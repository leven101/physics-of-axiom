	
\section{Introduction}
Axioms are definitions that are considered true a priori. Axioms are defined as already true based on the inability to deny their existence. The question I am interested to answer is how this perception is achieved as a logical function of reality. 

Definitions and axioms are the basis on which all of mathematics is constructed. Mathematical models define the truth of the things in the universe we can define. Given this, axiom may seem like too large a topic to tackle especially from the perspective of physical definition. It could be argued that this is not the realm of hard science but a philosophical quest. However a short explanation of the pathway to my reasoning may explain some of my undoubtedly out of the box conclusions on this topic. 

\subsection{About myself}
My interest in this area grew from my private study of popular theories of quantum physics aimed at describing a theory of everything. The standard model of particle physics (or equally any quantum interpretation of it) aim is to discover the categories of the energies that our universe is composed of. From a theoretical mathematical point of view as opposed to a physical one a theory of everything equates to a framework for a definition --- not necessarily a solution --- of all possible functions. All mathematics are based off of the axioms which, I will show, have no satisfactory definition currently. Therefore I aim to construct a more complete physical definition of axiom with the aim that it can be used to discover new physical properties. 

To understand my attempt of a physical interpretation  of definition I will have to give a brief introduction of myself. My academic background and work experience are in the area of computer science and artificial intelligence (AI).  I have never studied physics at any level and have never done academic research in the area. Privately I have studied particle physics and alternative quantum theories to some degree. I am good at self learning. I had no formal education prior to university and am largely self taught. This included no education on the hard sciences of math, chemistry or physics. No other subject was taught either. Now I have some academic credentials. In 2011 I was awarded my PhD in Computer science and AI from the School of Informatics, University of Edinburgh. In addition I was employed as an statistical AI researcher at the University of Oxford for four further years. Now I am VP, Data Science for WorkFusion, a successful AI start up based on Wall Street. My interest in a more rigorous concept of definition grew  during my time at the University of Oxford while reading about theoretical physics as a hobby. 

\subsection{Theory of Everything vs Axiomatic Definition}

As I said, my interest in particle physics is purely incidental and all my study has primarily been from the perspective of CS and big data. Though there are a large number of theories targeting a more complete definition of energy in the universe including string theory, m-theory, wave-particle duality, loop quantum gravity, dark energy, dark matter, etc. \footnote{A up to date interactive summary of the theories in this space can be found at \url{https://www.quantamagazine.org/frontier-of-physics-interactive-map-20150803/}} in reality it is a problem that we are not yet close to solving. For example, during a talk I attended at the University of Oxford's Physics Department by scientists and engineers from the LHC at CERN the dearth of our current body of knowledge of the universe was impressed on the audience. Even after the discovery of the Higgs boson and incorporating it into the standard model the scientists explained they explain less than ten percent of matter in the universe that is observable via our telescopes and satellites. 

If we believe Albert Einstein then the difference between matter and energy is only the speed at which the composite particles interact. Speed up a table fast enough and you will have pure energy. Slow down pure energy enough and you might get a turtle shell. The fact that now scientists are unable to explain the bulk of the physical matter in the universe implies strongly we must not have yet discovered some categories of energy critical to the structure of the matter we can observe. 

\begin{comment}
As I began to think about the problem more I was struck by the religious connotations inherent in a theory of everything. Beyond a purely semantic logical inclusion ? a theory of everything should include religion ? the problem of course strikes at the core of ancient idea about God. My parents were converted Jews and I was born and raised in an independent religious sect. As devout Christians, my parents raised me to believe that God was an all powerful consciousness. God?s definition was an eternal energy that was at all times omnipresent, in every possible place at the same time, omniscient, contained all knowledge, and omnipotent, an all powerful creature. If I were religious I would in essence by exploring theories of my childhood idea of a God. However I am not religious by nature and in any case God is hardly considered a satisfying and complete scientific solution. 

In lieu of relying on God, or gods, humans have satisfied our endless curiosity using numbers and functions.
\end{comment}

Throughout history humans have satisfied our endless curiosity about the phenomenon that surrounds us using numbers and functions. Numbers are the most primitive functions that are also axioms. There is a definition for the number one but not a proof. Of course all proofs are based off of the axioms of numbers. Being a computer scientist I often use the number zero to begin any counting sequence. Starting with zero we could consider a theory of everything as the inverse of the definition of nothing. Nothing is defined as a type of singularity that contains only itself. A complete theory of everything should contain in it a definition of nothing plus everything else, including itself. 

The idea of notational self contained completeness is of course a well studied phenomenon with my famous results. For examples, 
\begin{itemize}
\item Incompleteness theorem essentially proves that there cannot exist an annotation (symbolic logical language) such that it is fully self contained and can also prove all axioms of the logic without contradiction. 
\item Russell's paradox shows a logical contradiction in the formulation of any set said to contain all sets. 
\end{itemize}

These theories negate the possibility for a complete and logical theory of everything using a single self contained notational language. The search for a theory of everything however is not limited by notational theory and logic. Physicists and mathematicians are searching for a coherent set of theoretical components that provides a framework that is capable of encapsulating all known phenomena in the universe. These theoretical sub-components are not restricted to a unified notation. However all the theoretical components have their basis on geometric axioms. The concept of axiom itself has not until now been given any physical or geometric interpretation. This is the aim of this treatise. 
